\documentclass{article}
\usepackage{geometry}
\usepackage{graphicx, url, hyperref, amsmath, amssymb, mathtools, comment, xcolor}
\usepackage{amsthm}
\usepackage[most]{tcolorbox}
\tcbuselibrary{skins, breakable}
\usepackage[shortlabels]{enumitem}
\usepackage{mathscinet} % for \Dbar
\usepackage{caption}
\usepackage{booktabs}
\usepackage{tabularx}

% Geometry settings
\geometry{margin=1.5in}

% --- Theorems and Styles ---
\theoremstyle{plain}
\newtheorem{theorem}{Theorem}
\newtheorem{lemma}{Lemma}
\newtheorem{claim}{Claim}

\theoremstyle{definition}
\newtheorem*{example}{Example}
\newtheorem{remark}{Remark}[section]
\newtheorem{definition}[theorem]{Definition}

% --- Custom Commands ---
\newcommand{\Aletheia}{\emph{Aletheia}}
\renewcommand{\comment}[1]{\marginpar{{\tiny{#1}\normalfont\par}}}
\newcommand{\tony}[1]{{\color{red}Tony: #1}}

% --- Defined Colors ---
\definecolor{boxblue}{RGB}{0, 0, 150}
\definecolor{boxback}{RGB}{245, 245, 255}

% --- The Problem Environment ---
\newtcolorbox{problem}[1]{%
    colback=boxback,
    colframe=boxblue,
    fonttitle=\bfseries\large,
    title={#1},
    sharp corners,
    enhanced,
    attach boxed title to top left={yshift=-2mm, xshift=2mm},
    boxed title style={colframe=boxblue, colback=boxblue},
    before skip=15pt plus 2pt,
    after skip=15pt plus 2pt,
    top=10pt, bottom=10pt, left=10pt, right=10pt
}

\newtcolorbox{solution}[1]{%
    colback=white,
    colframe=boxblue,
    fonttitle=\bfseries\large,
    title={#1},
    sharp corners,
    enhanced jigsaw, % Better frame handling for page breaks than just 'enhanced'
    breakable,       % <--- Allows the box to split across pages
    attach boxed title to top left={yshift=-2mm, xshift=2mm},
    boxed title style={colframe=boxblue, colback=boxblue},
    before skip=15pt plus 2pt,
    after skip=15pt plus 2pt,
    top=10pt, bottom=10pt, left=10pt, right=10pt
}


\date{}


\begin{document}
\emph{The prompts and responses below have been edited only for
formatting compatibility.}

\begin{problem}{FirstProof \# 9}Let $n \geq 5$.
Let $A^{(1)}, \ldots, A^{(n)} \in \mathbb{R}^{3 \times 4}$ be Zariski-generic.
For $\alpha, \beta, \gamma, \delta \in [n]$, construct $Q^{(\alpha \beta \gamma \delta)} \in \mathbb{R}^{3 \times 3 \times 3 \times 3}$ so that its $(i, j, k, \ell)$ entry for $1 \leq i, j, k, \ell \leq 3$ is given by $Q^{(\alpha \beta \gamma \delta)}_{i j k \ell} = \det [A^{(\alpha)}(i, :); A^{(\beta)}(j, :); A^{(\gamma)}(k, :); A^{(\delta)}(\ell, :)]$.
Here $A(i, :)$ denotes the $i$th row of a matrix $A$, and semicolon denotes vertical concatenation.
We are interested in algebraic relations on the set of tensors $\{Q^{(\alpha \beta \gamma \delta)} : \alpha, \beta, \gamma, \delta \in [n] \}$.

More precisely, does there exist a polynomial map $\mathbf{F}: \mathbb{R}^{81n^4} \rightarrow \mathbb{R}^N$ that satisfies the following three properties?
\smallskip
\begin{itemize}\setlength\itemsep{0.5em}
\item The map $\mathbf{F}$ does not depend on $A^{(1)}, \ldots A^{(n)}$.
\item The degrees of the coordinate functions of $\mathbf{F}$ do not depend on $n$.
\item Let $\lambda \in \mathbb{R}^{n \times n \times n \times n}$ satisfy
$\lambda_{\alpha \beta \gamma \delta} \neq 0$ for precisely $\alpha, \beta, \gamma, \delta \in [n]$ that are not identical.  Then $\mathbf{F}(\lambda_{\alpha \beta \gamma \delta} Q^{(\alpha \beta \gamma \delta)} : \alpha, \beta, \gamma, \delta \in [n]) = 0$ holds if and only if there exist $u, v, w, x \in (\mathbb{R}^*)^n$ such that $\lambda_{\alpha \beta \gamma \delta} = u_{\alpha} v_{\beta} w_{\gamma} x_{\delta}$ for all $\alpha, \beta, \gamma, \delta \in [n]$ that are not identical.
\end{itemize}
\end{problem}

\begin{solution}{Model Response (Agent A)}
    \section*{1. Construction of the Polynomial Map $\mathbf{F}$}

We construct a polynomial map $\mathbf{F}: \mathbb{R}^{81n^4} \to \mathbb{R}^N$ whose coordinate functions are homogeneous polynomials of degree $3$.

Let $T$ be a formal tensor of variables in $\mathbb{R}^{n \times n \times n \times n \times 3 \times 3 \times 3 \times 3}$. The $81n^4$ entries of $T$, denoted $T^{(\alpha \beta \gamma \delta)}_{i j k \ell}$ for positional indices $\alpha, \beta, \gamma, \delta \in [n]$ and row choices $i, j, k, \ell \in \{1, 2, 3\}$, serve as the algebraic inputs to $\mathbf{F}$. We will subsequently evaluate $\mathbf{F}$ under the substitution $T^{(\alpha \beta \gamma \delta)}_{i j k \ell} = \lambda_{\alpha \beta \gamma \delta} Q^{(\alpha \beta \gamma \delta)}_{i j k \ell}$.

To build the coordinate functions of $\mathbf{F}$, we proceed as follows:
\begin{enumerate}
    \item Fix an ordered pair of distinct tensor modes (e.g., the third and fourth modes). Assign them strictly distinct positional indices $\gamma, \delta \in [n]$ with $\gamma \neq \delta$, alongside arbitrary row indices $k, \ell \in \{1, 2, 3\}$.
    \item For the remaining two varying modes (e.g., the first and second modes), choose three strictly distinct indices for the first mode: $\alpha_1 < \alpha_2 < \alpha_3$ in $[n]$, and three strictly distinct indices for the second mode: $\beta_1 < \beta_2 < \beta_3$ in $[n]$.
    \item Assign arbitrary row configurations $i_1, i_2, i_3 \in \{1, 2, 3\}$ and $j_1, j_2, j_3 \in \{1, 2, 3\}$.
\end{enumerate}

Construct a $3 \times 3$ matrix $B$ populated by the corresponding formal variables from $T$:
\[
B_{r, s} = T^{(\alpha_r \beta_s \gamma \delta)}_{i_r j_s k \ell} \quad \text{for } 1 \leq r, s \leq 3.
\]
We define one coordinate function of $\mathbf{F}$ to be the polynomial $\det(B)$. The complete map $\mathbf{F}$ is formed by collecting all such determinantal polynomials over all $\binom{4}{2} = 6$ valid pairs of varying modes, and over all valid combinatorial choices of positional and row indices.

\textbf{Constraint Verification:} By definition, each coordinate function evaluates a standard $3 \times 3$ minor of the formal variables. Its coefficients are exclusively $\pm 1$ or $0$. Thus, $\mathbf{F}$ depends entirely on combinatorial structures and is strictly independent of the entries of the generic matrices $A^{(m)}$. Furthermore, every coordinate function is a homogeneous polynomial of degree exactly $3$, satisfying the requirement that the degrees of the coordinate functions do not depend on $n$.

\section*{2. Proof of Sufficiency ($\impliedby$)}

Assume there exist parameters $u, v, w, x \in (\mathbb{R}^*)^n$ such that $\lambda_{\alpha \beta \gamma \delta} = u_\alpha v_\beta w_\gamma x_\delta$ precisely over the domain of tuples $(\alpha, \beta, \gamma, \delta)$ that are not uniformly identical.

Evaluating $\mathbf{F}$ at $T = \lambda \odot Q$ (where $\odot$ denotes the pointwise Hadamard product), the matrix $B$ constructed above via the first and second varying modes takes the form:
\[
B_{r, s} = \lambda_{\alpha_r \beta_s \gamma \delta} Q^{(\alpha_r \beta_s \gamma \delta)}_{i_r j_s k \ell}.
\]
Because the construction strictly enforced $\gamma \neq \delta$, the evaluated tuple $(\alpha_r, \beta_s, \gamma, \delta)$ fundamentally avoids identical uniformity. Hence, substituting the rank-$1$ factorization yields:
\[
B_{r, s} = u_{\alpha_r} v_{\beta_s} w_\gamma x_\delta \det \big[ A^{(\alpha_r)}(i_r, :); A^{(\beta_s)}(j_s, :); A^{(\gamma)}(k, :); A^{(\delta)}(\ell, :) \big].
\]
This factors the matrix as $B = c \cdot D_u \widetilde{Q} D_v$, where $c = w_\gamma x_\delta \neq 0$, the diagonal scaling matrices $D_u = \operatorname{diag}(u_{\alpha_1}, u_{\alpha_2}, u_{\alpha_3})$ and $D_v = \operatorname{diag}(v_{\beta_1}, v_{\beta_2}, v_{\beta_3})$ are non-singular, and $\widetilde{Q}$ is the core multilinear $4 \times 4$ determinant matrix defined by
\[
\widetilde{Q}_{r, s} = \det \big[ A^{(\alpha_r)}(i_r, :); A^{(\beta_s)}(j_s, :); A^{(\gamma)}(k, :); A^{(\delta)}(\ell, :) \big].
\]

Because the matrices $A^{(m)}$ are Zariski-generic and $\gamma \neq \delta$, the selected fixed rows $A^{(\gamma)}(k, :)$ and $A^{(\delta)}(\ell, :)$ are linearly independent and naturally span a $2$-dimensional subspace $V \subset \mathbb{R}^4$. The $4 \times 4$ determinant evaluates the standard volume form on $\mathbb{R}^4$, which mathematically descends to an alternating bilinear form on the $2$-dimensional quotient space $\mathbb{R}^4 / V \cong \mathbb{R}^2$. Letting $\pi: \mathbb{R}^4 \to \mathbb{R}^4 / V$ be the canonical projection, there exists a non-zero scalar $C$ (dependent strictly on $V$) such that:
\[
\widetilde{Q}_{r, s} = C \det \Big( \pi \big(A^{(\alpha_r)}(i_r, :)\big), \pi \big(A^{(\beta_s)}(j_s, :)\big) \Big),
\]
where the outer determinant represents the canonical $2 \times 2$ volume form on the quotient space. Algebraically, this structural equivalence dictates that the $3 \times 3$ matrix $\widetilde{Q}$ factors precisely through an intermediate $2$-dimensional vector space. Consequently, $\operatorname{rank}(\widetilde{Q}) \leq 2$. Thus, all evaluated $3 \times 3$ minors identically vanish, forcing $\det(B) = 0$ universally across all valid mode and index selections. This establishes sufficiency.

\section*{3. Proof of Necessity ($\implies$): Local Rank-1 Separability}

Assume $\mathbf{F}(\lambda \odot Q) \equiv 0$. Expanding the constraint $\det(B) = 0$ via the Leibniz formula yields:
\[
\sum_{\sigma \in \mathfrak{S}_3} \operatorname{sgn}(\sigma) \prod_{r=1}^3 \Big( \lambda_{\alpha_r \beta_{\sigma(r)} \gamma \delta} \widetilde{Q}_{r, \sigma(r)} \Big) = 0.
\]
Let $v_{r} = \pi(A^{(\alpha_r)}(i_r, :))$ and $u_{s} = \pi(A^{(\beta_s)}(j_s, :))$. Factoring out the volume scalar $C^3 \neq 0$ and denoting the coefficient component by $W_\sigma = \prod_{r=1}^3 \lambda_{\alpha_r \beta_{\sigma(r)} \gamma \delta}$, the sum defines a multilinear map $L$ evaluating to zero:
\[
L(v_1, v_2, v_3, u_1, u_2, u_3) = \sum_{\sigma \in \mathfrak{S}_3} \operatorname{sgn}(\sigma) W_\sigma \prod_{r=1}^3 \det(v_r, u_{\sigma(r)}) = 0.
\]

For each fixed index $r \in \{1, 2, 3\}$, the set of possible values for the vector $v_r$ is generated by independently varying the assigned row choice $i_r \in \{1, 2, 3\}$. Because each single matrix $A^{(\alpha_r)}$ is Zariski-generic, its three available rows span a $3$-dimensional subspace of $\mathbb{R}^4$. The intersection of this $3$-dimensional subspace with the $2$-dimensional kernel $V$ of the projection $\pi$ is generically $1$-dimensional. Consequently, as $i_r$ varies, the set of three available projections $\{\pi(A^{(\alpha_r)}(1, :)), \pi(A^{(\alpha_r)}(2, :)), \pi(A^{(\alpha_r)}(3, :))\}$ dynamically spans the entirety of the $2$-dimensional target space $\mathbb{R}^4 / V \cong \mathbb{R}^2$.

Since this independent spanning property holds universally for the domain of every argument slot $v_r$ as $i_r$ varies (and symmetrically for $u_s$ as $j_s$ varies), the multilinear map $L$ evaluates directly to zero on the full Cartesian product of these structurally full independent spanning sets. Standard multilinear algebra therefore rigorously dictates that $L \equiv 0$ identically everywhere on $(\mathbb{R}^2)^6$.

By the First and Second Fundamental Theorems of Invariant Theory for $\mathrm{SL}(2, \mathbb{R})$ (see H. Weyl, \textit{The Classical Groups: Their Invariants and Representations}, Princeton University Press, 1939), the vector space of multilinear $\mathrm{SL}(2, \mathbb{R})$-invariants on six generic vectors in $\mathbb{R}^2$ has dimension exactly equal to the Catalan number $C_3 = 5$. There are precisely $3! = 6$ uniquely generated multi-determinant bipartite invariants $P_\sigma = \prod_{r=1}^3 \det(v_r, u_{\sigma(r)})$. Because these bipartite invariants span the entire $5$-dimensional invariant space (linearly generating all non-crossing perfect matchings via successive Grassmann-Plücker syzygies), the space of linear relations strictly spanning them is exactly $1$-dimensional ($6 - 5 = 1$). This unique relation is universally generated by the primary Grassmann-Plücker identity:
\[
\sum_{\sigma \in \mathfrak{S}_3} \operatorname{sgn}(\sigma) P_\sigma = 0.
\]

Because the syzygy space is strictly $1$-dimensional, the identically vanishing linear combination $\sum_{\sigma \in \mathfrak{S}_3} \operatorname{sgn}(\sigma) W_\sigma P_\sigma \equiv 0$ functionally mandates that the coefficients $\operatorname{sgn}(\sigma) W_\sigma$ are strictly proportional to the syzygy coefficients $\operatorname{sgn}(\sigma)$. This systematically yields $W_\sigma = K$, where $K$ is a constant completely independent of $\sigma$. Equating $W_{\operatorname{id}} = W_{(1 2)}$ systematically yields:
\[
\lambda_{\alpha_1 \beta_1 \gamma \delta} \lambda_{\alpha_2 \beta_2 \gamma \delta} \lambda_{\alpha_3 \beta_3 \gamma \delta} = \lambda_{\alpha_1 \beta_2 \gamma \delta} \lambda_{\alpha_2 \beta_1 \gamma \delta} \lambda_{\alpha_3 \beta_3 \gamma \delta}.
\]
Because $\gamma \neq \delta$, the tuple $(\alpha_3, \beta_3, \gamma, \delta)$ categorically avoids identical uniformity. By hypothesis, $\lambda$ strictly vanishes exclusively on uniformly identical tuples, reliably guaranteeing $\lambda_{\alpha_3 \beta_3 \gamma \delta} \neq 0$. Factoring it out isolates the local $2 \times 2$ rank-$1$ relation over the varying domains:
\[
\lambda_{\alpha_1 \beta_1 \gamma \delta} \lambda_{\alpha_2 \beta_2 \gamma \delta} = \lambda_{\alpha_1 \beta_2 \gamma \delta} \lambda_{\alpha_2 \beta_1 \gamma \delta}.
\]

\section*{4. Global Rank-1 Gluing (Necessity)}

We now uncouple the variables and bridge domains to synthesize the local rank-$1$ conditions into a global factorization. To streamline notation, we re-index the tensor positional modes as $(a, b, c, d) \in [n]^4$. Let $\Omega \subset [n]^4$ be the domain of non-identical tuples. By hypothesis, $\lambda_{abcd} \neq 0$ for all $(a, b, c, d) \in \Omega$.

Fix an arbitrary index $a_0 \in [n]$. Consider the normalized ratio:
\[
R(a \mid b, c, d) = \frac{\lambda_{a b c d}}{\lambda_{a_0 b c d}}.
\]
This ratio is universally well-defined provided $(a_0, b, c, d) \in \Omega$, which holds unconditionally whenever $b, c, d$ are not all simultaneously equal to $a_0$.

By symmetrically evaluating the local rank-$1$ relation across different pairs of varying modes, we deduce the following uncoupled constraints:
\begin{enumerate}
    \item When $c \neq d$, the local rank-$1$ factorization (derived by varying the first and second modes) implies the submatrix $M^{(cd)}_{a,b} = \lambda_{abcd}$ natively has rank $1$. Thus, $\lambda_{a b c d} = \tilde{A}_{c,d}(a) \tilde{B}_{c,d}(b)$. Therefore, the ratio $R(a \mid b, c, d) = \tilde{A}_{c,d}(a)/\tilde{A}_{c,d}(a_0)$ unconditionally filters out the variable $b$, allowing us to safely denote it as a function $P(a, c, d)$.
    \item When $b \neq d$, by symmetric application (varying the first and third modes), the ratio strictly uncouples independently of the variable $c$, yielding a function $Q(a, b, d)$.
    \item When $b \neq c$, by symmetry (varying the first and fourth modes), the ratio strictly uncouples independently of the variable $d$, yielding a function $S(a, b, c)$.
\end{enumerate}

For any tuple satisfying both $c \neq d$ and $b \neq d$, we have $P(a, c, d) = R(a \mid b, c, d) = Q(a, b, d)$. Because $P$ is independent of $b$ and $Q$ is independent of $c$, their common structural value must be functionally independent of both $b$ and $c$, strictly reducing to a function $U(a, d)$. For any fixed pair $c \neq d$, since $n \geq 5$, we can fundamentally select an index $b \notin \{c, d, a_0\}$. For this choice of $b$, the conditions $(a_0, b, c, d) \in \Omega$, $c \neq d$, and $b \neq d$ hold simultaneously. This mathematically forces $P(a, c, d) = U(a, d)$ identically for all $c \neq d$.

Similarly, on the intersection spanning $c \neq d$ and $b \neq c$, overlapping yields $U(a, d) = S(a, b, c)$. By an analogous valid index selection for $c$, this structurally isolates the variables completely independent of $d$. Therefore, $U$ inherently restricts to a univariate global function dependent solely on $a$: $U(a, d) \equiv u_1(a)$. Consequently, $P(a, c, d) = u_1(a)$ for all $c \neq d$, meaning $R(a \mid b, c, d) = u_1(a)$ universally over the subspace where $c \neq d$.

This establishes domain relations seamlessly across three operative partitions of $\Omega$:
\begin{itemize}
    \item $D_1 = \{ (a,b,c,d) \in \Omega \mid c \neq d \}$: $R = u_1(a) \implies \lambda_{a b c d} = u_1(a) \lambda_{a_0 b c d} = u_1(a) F_1(b, c, d)$. Since $M^{(cd)}_{a,b}$ has rank $1$, $F_1$ must factor out the dependence on $b$. By defining an analogous ratio for the second mode and applying identical symmetric reasoning, we globally extract a univariate function $u_2(b)$. Thus, the relation uncouples strictly into $\lambda_{a b c d} = u_1(a) u_2(b) G_1(c, d)$.
    \item $D_2 = \{ (a,b,c,d) \in \Omega \mid a \neq b \}$: By complete corresponding symmetry, this structurally uncouples into $\lambda_{a b c d} = u_3(c) u_4(d) G_2(a, b)$, extracting global univariate functions $u_3(c)$ and $u_4(d)$.
    \item $D_3 = \{ (a,b,c,d) \in \Omega \mid b \neq d \}$: Symmetrically, the variables uniquely uncouple to $\lambda_{a b c d} = u_1'(a) u_3'(c) G_3(b, d)$. Overlapping exactly with $D_1$ and $D_2$ logically guarantees $u_1'$ and $u_3'$ correspond directly (up to constant scaling) to $u_1$ and $u_3$, yielding $\lambda_{a b c d} = u_1(a) u_3(c) G_3(b, d)$.
\end{itemize}

On the intersection bridging $D_1 \cap D_2 = \{ c \neq d \text{ and } a \neq b \}$, evaluation strictly crosses conditions:
\[
u_1(a) u_2(b) G_1(c, d) = u_3(c) u_4(d) G_2(a, b).
\]
Because $\lambda$ categorically avoids $0$ on $\Omega$, the functional components are identically non-zero. This cross-condition fundamentally separates the localized functions, forcing the fraction $\frac{G_1(c, d)}{u_3(c) u_4(d)} = \frac{G_2(a, b)}{u_1(a) u_2(b)} = K$ for some strictly non-zero constant $K$. Thus $G_1(c, d) = K u_3(c) u_4(d)$. For any given pair $(c,d) \in D_1$, we can select $a, b$ such that $a \neq b$, implying $(a,b,c,d) \in D_1 \cap D_2$. Thus $G_1(c,d) = K u_3(c) u_4(d)$ holds identically everywhere on $D_1$. Absorbing $K$ structurally into $u_4(d)$ seamlessly unifies the mapping precisely on $D_1 \cup D_2$:
\[
\lambda_{a b c d} = u_1(a) u_2(b) u_3(c) u_4(d).
\]

Validating continuity onto the third uncoupled plane naturally extends into $D_3$. On $(D_1 \cup D_2) \cap D_3$:
\[
u_1(a) u_3(c) G_3(b, d) = u_1(a) u_2(b) u_3(c) u_4(d) \implies G_3(b, d) = u_2(b) u_4(d).
\]
Because for any $(b,d) \in D_3$, we can reliably select an index $c$ such that $c \neq d$, every element mapping in $D_3$ securely lies in $D_1 \cap D_3$. This definitively anchors the exact global rank-$1$ factorization over the united domain $D_1 \cup D_2 \cup D_3$.

\textbf{Completeness of Domain Coverage:} If any valid tuple $(a,b,c,d) \in \Omega$ were to fall outside the spanned domain $D_1 \cup D_2 \cup D_3$, its exact mathematical omission logically structures the rigorous constraints: $c = d$ (violating $D_1$), $a = b$ (violating $D_2$), and $b = d$ (violating $D_3$). These three constraints collectively restrict the index boundaries exclusively to $a = b = c = d$, fully characterizing identical uniformity. However, uniformly identical tuples are unconditionally excluded from $\Omega$ strictly by hypothesis.

Thus, the global union $D_1 \cup D_2 \cup D_3$ perfectly equals $\Omega$, entirely covering all valid non-identical subsets and firmly securing the exact uncoupled relation $\lambda_{\alpha \beta \gamma \delta} = u_1(\alpha) u_2(\beta) u_3(\gamma) u_4(\delta)$. Concurrently, since structural evaluation over $\Omega$ categorically rejects evaluation to zero, it unequivocally enforces the required continuous functional constraints $u_1, u_2, u_3, u_4 \in (\mathbb{R}^*)^n$ (formally corresponding to the vectors $u, v, w, x$). This comprehensively establishes necessity and completes the full proof.
\end{solution}

\end{document}
