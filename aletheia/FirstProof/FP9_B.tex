\documentclass{article}
\usepackage{geometry}
\usepackage{graphicx, url, hyperref, amsmath, amssymb, mathtools, comment, xcolor}
\usepackage{amsthm}
\usepackage[most]{tcolorbox}
\tcbuselibrary{skins, breakable}
\usepackage[shortlabels]{enumitem}
\usepackage{mathscinet} % for \Dbar
\usepackage{caption}
\usepackage{booktabs}
\usepackage{tabularx}

% Geometry settings
\geometry{margin=1.5in}

% --- Theorems and Styles ---
\theoremstyle{plain}
\newtheorem{theorem}{Theorem}
\newtheorem{lemma}{Lemma}
\newtheorem{claim}{Claim}

\theoremstyle{definition}
\newtheorem*{example}{Example}
\newtheorem{remark}{Remark}[section]
\newtheorem{definition}[theorem]{Definition}

% --- Custom Commands ---
\newcommand{\Aletheia}{\emph{Aletheia}}
\renewcommand{\comment}[1]{\marginpar{{\tiny{#1}\normalfont\par}}}
\newcommand{\tony}[1]{{\color{red}Tony: #1}}

% --- Defined Colors ---
\definecolor{boxblue}{RGB}{0, 0, 150}
\definecolor{boxback}{RGB}{245, 245, 255}

% --- The Problem Environment ---
\newtcolorbox{problem}[1]{%
    colback=boxback,
    colframe=boxblue,
    fonttitle=\bfseries\large,
    title={#1},
    sharp corners,
    enhanced,
    attach boxed title to top left={yshift=-2mm, xshift=2mm},
    boxed title style={colframe=boxblue, colback=boxblue},
    before skip=15pt plus 2pt,
    after skip=15pt plus 2pt,
    top=10pt, bottom=10pt, left=10pt, right=10pt
}

\newtcolorbox{solution}[1]{%
    colback=white,
    colframe=boxblue,
    fonttitle=\bfseries\large,
    title={#1},
    sharp corners,
    enhanced jigsaw, % Better frame handling for page breaks than just 'enhanced'
    breakable,       % <--- Allows the box to split across pages
    attach boxed title to top left={yshift=-2mm, xshift=2mm},
    boxed title style={colframe=boxblue, colback=boxblue},
    before skip=15pt plus 2pt,
    after skip=15pt plus 2pt,
    top=10pt, bottom=10pt, left=10pt, right=10pt
}


\date{}


\begin{document}
\emph{The prompts and responses below have been edited only for
formatting compatibility (to compile in a \LaTeX document).}

\begin{problem}{FirstProof \# 9}Let $n \geq 5$.
Let $A^{(1)}, \ldots, A^{(n)} \in \mathbb{R}^{3 \times 4}$ be Zariski-generic.
For $\alpha, \beta, \gamma, \delta \in [n]$, construct $Q^{(\alpha \beta \gamma \delta)} \in \mathbb{R}^{3 \times 3 \times 3 \times 3}$ so that its $(i, j, k, \ell)$ entry for $1 \leq i, j, k, \ell \leq 3$ is given by $Q^{(\alpha \beta \gamma \delta)}_{i j k \ell} = \det [A^{(\alpha)}(i, :); A^{(\beta)}(j, :); A^{(\gamma)}(k, :); A^{(\delta)}(\ell, :)]$.
Here $A(i, :)$ denotes the $i$th row of a matrix $A$, and semicolon denotes vertical concatenation.
We are interested in algebraic relations on the set of tensors $\{Q^{(\alpha \beta \gamma \delta)} : \alpha, \beta, \gamma, \delta \in [n] \}$.

More precisely, does there exist a polynomial map $\mathbf{F}: \mathbb{R}^{81n^4} \rightarrow \mathbb{R}^N$ that satisfies the following three properties?
\smallskip
\begin{itemize}\setlength\itemsep{0.5em}
\item The map $\mathbf{F}$ does not depend on $A^{(1)}, \ldots A^{(n)}$.
\item The degrees of the coordinate functions of $\mathbf{F}$ do not depend on $n$.
\item Let $\lambda \in \mathbb{R}^{n \times n \times n \times n}$ satisfy
$\lambda_{\alpha \beta \gamma \delta} \neq 0$ for precisely $\alpha, \beta, \gamma, \delta \in [n]$ that are not identical.  Then $\mathbf{F}(\lambda_{\alpha \beta \gamma \delta} Q^{(\alpha \beta \gamma \delta)} : \alpha, \beta, \gamma, \delta \in [n]) = 0$ holds if and only if there exist $u, v, w, x \in (\mathbb{R}^*)^n$ such that $\lambda_{\alpha \beta \gamma \delta} = u_{\alpha} v_{\beta} w_{\gamma} x_{\delta}$ for all $\alpha, \beta, \gamma, \delta \in [n]$ that are not identical.
\end{itemize}
\end{problem}

\begin{solution}{Model Response (Agent B)}

\section*{1. Construction of the Polynomial Map $\mathbf{F}$}

Let the input to our polynomial map $\mathbf{F}: \mathbb{R}^{81n^4} \rightarrow \mathbb{R}^N$ be the formal tensor variables $X$, mapping to the scaled quadrifocal tensor:
\[
X^{(\alpha \beta \gamma \delta)}_{i j k \ell} = \lambda_{\alpha \beta \gamma \delta} \det \begin{bmatrix} A^{(\alpha)}(i, :) \\ A^{(\beta)}(j, :) \\ A^{(\gamma)}(k, :) \\ A^{(\delta)}(\ell, :) \end{bmatrix},
\]
where $A^{(\alpha)}(i, :) \in \mathbb{R}^{1 \times 4}$ denotes the $i$-th row of the $3 \times 4$ Zariski-generic matrix $A^{(\alpha)}$.

There are $\binom{4}{2} = 6$ symmetric ways to partition the four tensor slots into two evaluated slots and two fixed slots. Consider the bipartition where we evaluate slots 1 and 2, and fix slots 3 and 4. For any pair of fixed matrix indices $\gamma \neq \delta \in [n]$ and fixed row indices $k, \ell \in \{1, 2, 3\}$, we define a $3n \times 3n$ evaluation matrix $\mathbf{M}^{\gamma, \delta}_{k, \ell}(X)$. Its rows are indexed by the multi-index $(\alpha, i) \in [n] \times \{1, 2, 3\}$ and its columns by $(\beta, j) \in [n] \times \{1, 2, 3\}$. The entries are identically the formal variables:
\[
\left( \mathbf{M}^{\gamma, \delta}_{k, \ell}(X) \right)_{(\alpha, i), (\beta, j)} = X^{(\alpha \beta \gamma \delta)}_{i j k \ell}.
\]

\textbf{Definition of $\mathbf{F}$:} We systematically define the coordinate functions of $\mathbf{F}$ to be all $3 \times 3$ minors of these $3n \times 3n$ evaluation matrices, comprehensively collected across all $6$ possible slot bipartitions and all valid choices of the fixed indices (strictly restricting to pairs where the two fixed generic matrix slots are distinct).

\textbf{Verification of Properties 1 and 2:}
\begin{enumerate}
    \item Because the entries of the evaluation matrices are precisely the raw formal variables $X$, the coordinate functions are standard $3 \times 3$ determinantal expansions. Their coefficients are strictly combinatorial constants ($\pm 1$ and $0$). Therefore, the map $\mathbf{F}$ does not depend on the specific entries of the generic matrices $A^{(1)}, \dots, A^{(n)}$.
    \item Every coordinate function is a $3 \times 3$ minor of degree-$1$ variables, naturally forming a homogeneous polynomial of exact degree $3$. This uniform degree structurally does not depend on $n$.
\end{enumerate}

\section*{2. Sufficiency: Rank-1 Scaling Implies $\mathbf{F}(X) = 0$}

Assume there exist vectors $u, v, w, x \in (\mathbb{R}^*)^n$ such that $\lambda_{\alpha \beta \gamma \delta} = u_\alpha v_\beta w_\gamma x_\delta$ for all non-identical tuples. When a tuple is completely identical (i.e., $\alpha=\beta=\gamma=\delta$), the unscaled determinant $Q^{(\alpha \alpha \alpha \alpha)}$ selects $4$ rows from the $3$ available rows of $A^{(\alpha)}$. By the pigeonhole principle, at least two rows must inherently repeat, unconditionally forcing $Q^{(\alpha \alpha \alpha \alpha)} = 0$. Thus, substituting the pure factorization $X^{(\alpha \beta \gamma \delta)}_{i j k \ell} = (u_\alpha v_\beta w_\gamma x_\delta) Q^{(\alpha \beta \gamma \delta)}_{i j k \ell}$ evaluates identically to $0=0$ on identical tuples, meaning the factorization representation holds universally.

Consider the $3n \times 3n$ evaluation matrix $\mathbf{M}(X) = \mathbf{M}^{\gamma, \delta}_{k, \ell}(X)$. Factoring out the global rank-$1$ scaling yields:
\[
\mathbf{M}(X) = (w_\gamma x_\delta) \mathbf{D}_{\mathbf{u}} \mathbf{M}(Q) \mathbf{D}_{\mathbf{v}},
\]
where $\mathbf{D}_{\mathbf{u}}$ and $\mathbf{D}_{\mathbf{v}}$ are block-diagonal scaling matrices, and $\mathbf{M}(Q)$ is the unscaled evaluation matrix.

Let $c = A^{(\gamma)}(k, :)^T$ and $d = A^{(\delta)}(\ell, :)^T$. The entries of $\mathbf{M}(Q)$ evaluate the alternating bilinear form $B(y, z) = \det(y, z, c, d)$ on the vectors $y = A^{(\alpha)}(i, :)^T$ and $z = A^{(\beta)}(j, :)^T$. Because $c$ and $d$ naturally reside in the radical of $B$, the form fundamentally descends to an alternating bilinear form on the quotient space $\mathbb{R}^4 / \operatorname{span}(c, d)$. Since $c$ and $d$ are linearly independent (as $\gamma \neq \delta$), this quotient space has dimension $2$.

The matrix $\mathbf{M}(Q)$ is the Gram-like evaluation matrix of this alternating bilinear form acting on $3n$ specific vectors. Because it evaluates the form on the projections of these $3n$ vectors into the $2$-dimensional quotient space $\mathbb{R}^4 / \operatorname{span}(c, d)$, the matrix $\mathbf{M}(Q)$ inherently factors through this $2$-dimensional space. Thus, $\mathbf{M}(Q)$ is algebraically bounded to have rank at most $2$. Multiplying by diagonal scaling matrices securely preserves this upper rank bound, systematically ensuring $\operatorname{rank}(\mathbf{M}(X)) \le 2$. Therefore, all $3 \times 3$ minors of $\mathbf{M}(X)$ vanish identically. By symmetric extension across all $6$ bipartitions, $\mathbf{F}(X) = 0$.

\section*{3. Necessity: Global Rank Factorization Forces Local Rank-1 Property}

Conversely, assume $\mathbf{F}(X) = 0$. We will rigorously establish that fixing any two slots to distinct indices globally restricts the corresponding $n \times n$ matrix slice of $\lambda$ to be strictly rank-$1$.

Consider the bipartition evaluating slots 1 and 2. Fix slots 3 and 4 to arbitrary distinct indices $\gamma \neq \delta$, and fix the row indices to $k=\ell=1$. Let $c = A^{(\gamma)}(1, :)^T$ and $d = A^{(\delta)}(1, :)^T$. Because $\gamma \neq \delta$, $c$ and $d$ are independent vectors from different Zariski-generic matrices.

Let $V = \mathbb{R}^4 / \operatorname{span}(c, d)$ be the $2$-dimensional quotient space. The alternating form $\omega(y, z) = \det(y, z, c, d)$ defined on $\mathbb{R}^4 \times \mathbb{R}^4$ induces a non-degenerate symplectic form on $V$. We select a basis for $V$ such that $\omega$ is canonically represented by the standard symplectic matrix $J = \left(\begin{smallmatrix} 0 & 1 \\ -1 & 0 \end{smallmatrix}\right)$.

For each $\alpha \in [n]$, let $Y_\alpha \in \mathbb{R}^{3 \times 2}$ be the coordinate matrix whose $i$-th row is the transpose of the coordinate vector of the projection of $A^{(\alpha)}(i, :)^T$ onto $V$. This exact transposition ensures the block matrix multiplication validly computes the symplectic form evaluation $\bar{y}^T J \bar{z} = \omega(y, z)$. We define the global block-diagonal matrix $\mathbf{Y} = \operatorname{diag}(Y_1, \dots, Y_n) \in \mathbb{R}^{3n \times 2n}$.
\begin{itemize}
    \item For $\alpha \notin \{\gamma, \delta\}$, the $3$ rows of $A^{(\alpha)}$ are generic and strictly independent of $c$ and $d$. Their projections are generic vectors spanning $V$, guaranteeing $Y_\alpha$ natively possesses full column rank $2$.
    \item For $\alpha = \gamma$, the first row of $A^{(\gamma)}$ is exactly $c^T$, which projects to $0$. However, its second and third rows are generic and independent of $c$ and $d$ (since $\gamma \neq \delta$), so they perfectly project to a basis of $V$. Thus $Y_\gamma$ still retains full column rank $2$. Identical structural logic ensures $Y_\delta$ retains full column rank $2$.
\end{itemize}
Because every individual $3 \times 2$ block $Y_\alpha$ possesses full column rank, the $3n \times 2n$ block-diagonal matrix $\mathbf{Y}$ unconditionally possesses full column rank $2n$. Hence, there exists an exact left-inverse matrix $\mathbf{L} \in \mathbb{R}^{2n \times 3n}$ such that $\mathbf{L}\mathbf{Y} = I_{2n}$.

Let $\Lambda \in \mathbb{R}^{n \times n}$ be the numeric matrix slice defined by $\Lambda_{\alpha, \beta} = \lambda_{\alpha \beta \gamma \delta}$. The scaled matrix $\mathbf{M}(X)$ evaluates to $\Lambda_{\alpha, \beta} Y_\alpha J Y_\beta^T$ on block $(\alpha, \beta)$, which perfectly factors globally as:
\[
\mathbf{M}(X) = \mathbf{Y} (\Lambda \otimes J) \mathbf{Y}^T.
\]
The assumption $\mathbf{F}(X) = 0$ requires that all $3 \times 3$ minors of $\mathbf{M}(X)$ vanish, algebraically bounding its rank to at most $2$. By multiplying by $\mathbf{L}$ on the left and $\mathbf{L}^T$ on the right, we flawlessly isolate the Kronecker core:
\[
\mathbf{L} \mathbf{M}(X) \mathbf{L}^T = \Lambda \otimes J.
\]
Since matrix multiplication cannot increase rank, $\operatorname{rank}(\Lambda \otimes J) \le \operatorname{rank}(\mathbf{M}(X)) \le 2$. By standard Kronecker product properties over any field, $\operatorname{rank}(\Lambda \otimes J) = \operatorname{rank}(\Lambda) \times \operatorname{rank}(J) = 2 \operatorname{rank}(\Lambda)$. This algebraically forces $2 \operatorname{rank}(\Lambda) \le 2$, unconditionally dictating $\operatorname{rank}(\Lambda) \le 1$.

Because $\gamma \neq \delta$, every tuple $(\alpha, \beta, \gamma, \delta)$ is strictly non-identical. By hypothesis, $\lambda$ evaluates to non-zero exactly on non-identical tuples, meaning $\Lambda$ contains absolutely no zero entries. As a rank-$1$ matrix devoid of zeros, $\Lambda$ factors exactly as $\Lambda_{\alpha, \beta} = U_\alpha V_\beta$ for non-zero vectors $U, V \in (\mathbb{R}^*)^n$.

\section*{4. Global Patching}

Since $n \ge 5$, we globally fix four mutually distinct anchor indices $a, b, c, d \in [n]$. Let $c_0 = \lambda_{a b c d} \neq 0$. We globally define four $1$-dimensional slice vectors:
\[
u_\alpha = \frac{\lambda_{\alpha b c d}}{c_0}, \quad v_\beta = \frac{\lambda_{a \beta c d}}{c_0}, \quad w_\gamma = \frac{\lambda_{a b \gamma d}}{c_0}, \quad x_\delta = \frac{\lambda_{a b c \delta}}{c_0}.
\]
Observe that evaluating at the anchors yields $u_a = v_b = w_c = x_d = 1$.

\textbf{Step 1: 2-Variable Base Slices.} By fixing two slots to distinct anchor indices, Section 3 guarantees the slice is rank-$1$. For example, fixing slots 3 and 4 to $c,d$ yields a slice rank-$1$ in $\alpha, \beta$, factoring as $U_\alpha V_\beta$. We evaluate:
\[
\frac{\lambda_{\alpha b c d} \lambda_{a \beta c d}}{\lambda_{a b c d}} = \frac{(U_\alpha V_b)(U_a V_\beta)}{U_a V_b} = U_\alpha V_\beta = \lambda_{\alpha \beta c d}.
\]
Substituting our defined vectors extracts the base slice $\lambda_{\alpha \beta c d} = \frac{(c_0 u_\alpha)(c_0 v_\beta)}{c_0} = c_0 u_\alpha v_\beta$. Applying this symmetrically extracts all six robust $2$-variable slices:
$\lambda_{\alpha \beta c d} = c_0 u_\alpha v_\beta$, $\lambda_{a b \gamma \delta} = c_0 w_\gamma x_\delta$, $\lambda_{\alpha b \gamma d} = c_0 u_\alpha w_\gamma$, $\lambda_{a \beta c \delta} = c_0 v_\beta x_\delta$, $\lambda_{\alpha b c \delta} = c_0 u_\alpha x_\delta$, and $\lambda_{a \beta \gamma d} = c_0 v_\beta w_\gamma$.

\textbf{Step 2: 3-Variable Lift.} Let $\gamma \neq \delta$. At least one inequality $\delta \neq b$ or $\gamma \neq b$ must strictly hold.
\begin{itemize}
\item \emph{Case 1 ($\delta \neq b$):} The slice $\lambda_{\cdot b \cdot \delta}$ securely fixes slots 2 and 4 to distinct indices, rendering it rank-$1$ in $\alpha, \gamma$. Using the Step 1 base slices, we safely interpolate:
\[
\lambda_{\alpha b \gamma \delta} = \frac{\lambda_{\alpha b c \delta} \lambda_{a b \gamma \delta}}{\lambda_{a b c \delta}} = \frac{(c_0 u_\alpha x_\delta)(c_0 w_\gamma x_\delta)}{c_0 x_\delta} = c_0 u_\alpha w_\gamma x_\delta.
\]
(The denominator evaluates to $\lambda_{a b c \delta}$; because anchors $a,b,c$ are mutually distinct, the tuple inherently contains at least $3$ distinct elements, strictly guaranteeing it is non-identical and safely bypassing any division by zero).
\item \emph{Case 2 ($\gamma \neq b$):} The slice $\lambda_{\cdot b \gamma \cdot}$ is rank-$1$ in $\alpha, \delta$. Interpolating symmetrically yields:
\[
\lambda_{\alpha b \gamma \delta} = \frac{\lambda_{\alpha b \gamma d} \lambda_{a b \gamma \delta}}{\lambda_{a b \gamma d}} = \frac{(c_0 u_\alpha w_\gamma)(c_0 w_\gamma x_\delta)}{c_0 w_\gamma} = c_0 u_\alpha w_\gamma x_\delta.
\]
\end{itemize}
Symmetrically, executing identical logic for the first slot rigorously establishes $\lambda_{a \beta \gamma \delta} = c_0 v_\beta w_\gamma x_\delta$ universally for all $\gamma \neq \delta$.

\textbf{Step 3: 4-Variable Lift.} On the core domain $\Omega_1 = \{(\alpha, \beta, \gamma, \delta) \mid \gamma \neq \delta\}$, the slice is rank-$1$ in $\alpha, \beta$. Substituting the $3$-variable cross-terms from Step 2:
\[
\lambda_{\alpha \beta \gamma \delta} = \frac{\lambda_{\alpha b \gamma \delta} \lambda_{a \beta \gamma \delta}}{\lambda_{a b \gamma \delta}} = \frac{(c_0 u_\alpha w_\gamma x_\delta)(c_0 v_\beta w_\gamma x_\delta)}{c_0 w_\gamma x_\delta} = c_0 u_\alpha v_\beta w_\gamma x_\delta.
\]
(Because $a \neq b$, the anchor tuple $(a,b,\gamma,\delta)$ is inherently non-identical, securely preventing any division by zero).

\textbf{Step 4: Set-Theoretic Exhaustive Covering.} By exploiting the intrinsic permutation symmetry of $\mathbf{F}$, applying the mathematically identical local rank-$1$ argument and global lifts to the bipartitions fixing slots 1,2 and slots 1,4 establishes the exact same factorization on the alternate domains $\Omega_2 = \{ (\alpha, \beta, \gamma, \delta) \mid \alpha \neq \beta \}$ and $\Omega_3 = \{ (\alpha, \beta, \gamma, \delta) \mid \alpha \neq \delta \}$.

The set-theoretic union $\Omega = \Omega_1 \cup \Omega_2 \cup \Omega_3$ mathematically covers every conceivable tuple configuration EXCEPT the state where $\gamma = \delta$, $\alpha = \beta$, and $\alpha = \delta$ occur simultaneously. Via transitivity, this uniquely isolates $\alpha = \beta = \gamma = \delta$ (the completely identical tuples). Because the hypothesis requires $\lambda$ to be non-zero exactly on non-identical tuples, every valid tuple intrinsically falls into the union $\Omega$.

By cleanly absorbing the non-zero constant $c_0$ into $u_\alpha$ (i.e., replacing $u_\alpha$ with $c_0 u_\alpha$), we achieve the exact outer product $\lambda_{\alpha \beta \gamma \delta} = u_\alpha v_\beta w_\gamma x_\delta$ universally over the non-identical support. Finally, because $\lambda$ is strictly non-zero on this support, the $1$-dimensional target vectors mathematically cannot house zero entries, securely guaranteeing $u, v, w, x \in (\mathbb{R}^*)^n$. This strictly fulfills Property 3 and decisively completes the proof.
\end{solution}

\end{document}
