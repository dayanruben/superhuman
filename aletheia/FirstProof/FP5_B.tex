\documentclass{article}
\usepackage{geometry}
\usepackage{graphicx, url, hyperref, amsmath, amssymb, mathtools, comment, xcolor}
\usepackage{amsthm}
\usepackage[most]{tcolorbox}
\tcbuselibrary{skins, breakable}
\usepackage[shortlabels]{enumitem}
\usepackage{mathscinet} % for \Dbar
\usepackage{caption}
\usepackage{booktabs}
\usepackage{tabularx}

% Geometry settings
\geometry{margin=1.5in}

% --- Theorems and Styles ---
\theoremstyle{plain}
\newtheorem{theorem}{Theorem}
\newtheorem{lemma}{Lemma}
\newtheorem{claim}{Claim}

\theoremstyle{definition}
\newtheorem*{example}{Example}
\newtheorem{remark}{Remark}[section]
\newtheorem{definition}[theorem]{Definition}

% --- Custom Commands ---
\newcommand{\Aletheia}{\emph{Aletheia}}
\renewcommand{\comment}[1]{\marginpar{{\tiny{#1}\normalfont\par}}}
\newcommand{\tony}[1]{{\color{red}Tony: #1}}

% --- Defined Colors ---
\definecolor{boxblue}{RGB}{0, 0, 150}
\definecolor{boxback}{RGB}{245, 245, 255}

% --- The Problem Environment ---
\newtcolorbox{problem}[1]{%
    colback=boxback,
    colframe=boxblue,
    fonttitle=\bfseries\large,
    title={#1},
    sharp corners,
    enhanced,
    attach boxed title to top left={yshift=-2mm, xshift=2mm},
    boxed title style={colframe=boxblue, colback=boxblue},
    before skip=15pt plus 2pt,
    after skip=15pt plus 2pt,
    top=10pt, bottom=10pt, left=10pt, right=10pt
}

\newtcolorbox{solution}[1]{%
    colback=white,
    colframe=boxblue,
    fonttitle=\bfseries\large,
    title={#1},
    sharp corners,
    enhanced jigsaw, % Better frame handling for page breaks than just 'enhanced'
    breakable,       % <--- Allows the box to split across pages
    attach boxed title to top left={yshift=-2mm, xshift=2mm},
    boxed title style={colframe=boxblue, colback=boxblue},
    before skip=15pt plus 2pt,
    after skip=15pt plus 2pt,
    top=10pt, bottom=10pt, left=10pt, right=10pt
}


\DeclareMathOperator{\vecop}{vec}
\DeclareMathOperator{\diag}{diag}
\DeclareMathAlphabet{\catsymbfont}{U}{rsfs}{m}{n}
\newcommand{\aA}{{\catsymbfont{A}}}

\newcommand{\bR}{\mathbb{R}}
\newcommand{\co}{\colon}
\newcommand{\scrS}{\mathscr{S}}
\newcommand{\aO}{{\catsymbfont{O}}}

\date{}


\begin{document}
\emph{The prompts and responses below have been edited only for
formatting compatibility (to compile in a \LaTeX document).}

\begin{problem}{FirstProof \# 5} Fix a finite group $G$.  Let $\aO$ denote an incomplete transfer
system associated to an $N_\infty$ operad.  Define the slice
filtration on the $G$-equivariant stable category adapted to $\aO$ and
state and prove a characterization of the $\aO$-slice connectivity of
a connective $G$-spectrum in terms of the geometric fixed points.
\end{problem}

\begin{solution}{Model Response (Agent B)}
The slice filtration provides a powerful structural filtration on the category of genuine $G$-spectra. The regular slice filtration has proven particularly advantageous because its slice connectivity can be characterized strictly in terms of geometric fixed points.

When working with $N_\infty$ operads, the permissible norm maps are governed by combinatorial structures defined on the poset of subgroups of $G$. In this paper, we construct a generalized slice filtration adapted to an arbitrary incomplete transfer system $\aO$. We then rigorously characterize its connective covers utilizing only the dimensions of these geometric fixed point spectra.

\section{Transfer Systems and $\aO$-Regular Representations}

Combinatorial models for $N_\infty$ operads inherently involve partial orders on the subgroups of $G$, formally known as \emph{transfer systems}. To align with the working context of our problem statement, we refer to them here as \emph{incomplete transfer systems}.

\begin{definition}
An incomplete transfer system $\aO$ on a finite group $G$ is a partial order $\to$ on the set of subgroups of $G$ satisfying the following axioms:
\begin{enumerate}
    \item \emph{(Refinement)} If $K \to H$, then $K \le H$.
    \item \emph{(Conjugation)} If $K \to H$, then $gKg^{-1} \to gHg^{-1}$ for all $g \in G$.
    \item \emph{(Restriction)} If $K \to H$ and $L \le H$, then $K \cap L \to L$.
\end{enumerate}
If $K \to H$, we say that $K$ is an $\aO$-subgroup of $H$.
\end{definition}

Transfer systems possess unique minimal elements, which are central to calculating our geometric connectivity bounds.

\begin{lemma} \label{lem:minimal_O_subgroup}
For any subgroup $H \le G$, the set of $\aO$-subgroups of $H$ is closed under intersection. Thus, there exists a unique minimal $\aO$-subgroup of $H$, which we denote by $K_H$. Furthermore, $K_H$ is a normal subgroup of $H$.
\end{lemma}
\begin{proof}
Let $A \to H$ and $B \to H$. Applying the restriction axiom to $A \to H$ and $B \le H$, we obtain $A \cap B \to B$. Since $B \to H$, the transitivity of the partial order yields $A \cap B \to H$. Because $H$ is finite and the partial order is reflexive (meaning the set of $\aO$-subgroups of $H$ is non-empty since $H \to H$), the intersection of all $\aO$-subgroups of $H$ is a well-defined $\aO$-subgroup, representing the unique minimum $K_H$. For any $h \in H$, the conjugation axiom implies $h K_H h^{-1} \to h H h^{-1} = H$. By minimality and uniqueness, $h K_H h^{-1} \ge K_H$, which forces $h K_H h^{-1} = K_H$ by cardinality considerations. Thus $K_H \unlhd H$.
\end{proof}

\begin{definition}
For each $H \le G$, define the \emph{$\aO$-regular representation} of $H$ to be the permutation representation $\rho_H^\aO = \mR[H/K_H]$. We define the \emph{$\aO$-index} of $H$ as $|H|_\aO = [H : K_H] = \dim_\mR(\rho_H^\aO)$.
\end{definition}

Following the approach used for the regular slice filtration, we define the $\aO$-slice filtration using these $\aO$-regular representations.

\begin{definition}
For $n \ge 0$, an \emph{$\aO$-slice cell of dimension $\ge n$} is a $G$-spectrum of the form $G_+ \wedge_H S^{m \rho_H^\aO - \epsilon}$, where $H \le G$, $m \in \mathbb{Z}_{\ge 0}$, $\epsilon \in \{0, 1\}$, and $m|H|_\aO - \epsilon \ge n$.
The category of $\aO$-slice $\ge n$ spectra, denoted $\tau_{\ge n}^\aO \SpG$, is the smallest full subcategory of the $G$-equivariant stable category containing all $\aO$-slice cells of dimension $\ge n$ that is closed under arbitrary wedges, extensions, positive suspensions, and retracts.
\end{definition}

\section{Geometric Fixed Point Characterization}

Recall that a non-equivariant spectrum $E$ is $c$-connective if $\pi_k(E) = 0$ for all $k < c$. A known result characterizes the regular slice filtration by showing $X \in \tau_{\ge n}$ if and only if $\Phi^L X$ is $(n/|L| - 1)$-connected. Noting that an $x$-connected spectrum is equivalently $\lfloor x \rfloor$-connected, which is in turn $(\lfloor x \rfloor + 1)$-connective, the condition naturally translates to being $\lfloor n/|L| \rfloor$-connective. We establish the precise analogue for an arbitrary incomplete transfer system $\aO$.

\begin{theorem} \label{thm:main}
Let $X$ be a connective $G$-spectrum and $n \ge 0$. Then $X \in \tau_{\ge n}^\aO \SpG$ if and only if for all subgroups $L \le G$, the geometric fixed point spectrum $\Phi^L X$ is $\lfloor n/|L|_\aO \rfloor$-connective.
\end{theorem}

The proof requires a sharp combinatorial lemma governing the $L$-fixed points of $\aO$-regular representations.

\begin{lemma} \label{lem:orbit_bound}
Let $H, L \le G$ and $g \in G$ be such that $L \le g H g^{-1}$. The number of $L$-orbits on the set $g H g^{-1} / g K_H g^{-1}$ under left multiplication is bounded below by $|H|_\aO / |L|_\aO$.
\end{lemma}
\begin{proof}
Let $H' = g H g^{-1}$ and $K' = g K_H g^{-1}$. By the conjugation axiom, $K' \to H'$. Since $K_H \unlhd H$ (Lemma \ref{lem:minimal_O_subgroup}), we have $K' \unlhd H'$. Thus $H'/K'$ is a well-defined group quotient, and $L \le H'$ acts on it by left multiplication. The $L$-orbits correspond to the right cosets of the image of $L$ inside $H'/K'$.

Because $K'$ is normal in $H'$, the stabilizer in $L$ of any coset $h'K'$ is precisely $L \cap h'K'(h')^{-1} = L \cap K'$. Therefore, every single $L$-orbit has the exact same uniform size: $w_L = [L : L \cap K']$. Applying the restriction axiom to $K' \to H'$ and $L \le H'$, we obtain $L \cap K' \to L$. Because $K_L$ is the unique minimal $\aO$-subgroup of $L$, it structurally follows that $K_L \le L \cap K'$. Thus, the uniform orbit size satisfies $w_L = [L : L \cap K'] \le [L : K_L] = |L|_\aO$.

The total number of $L$-orbits is exactly the cardinality of the underlying set divided by this uniform orbit size: $[H' : K'] / w_L = |H|_\aO / w_L \ge |H|_\aO / |L|_\aO$.
\end{proof}

\begin{proof}[Proof of Theorem \ref{thm:main}]
\textbf{Necessity ($\Rightarrow$):} Because the geometric fixed point functor $\Phi^L$ preserves exact triangles, positive suspensions, arbitrary wedges, and retracts, the full subcategory of connective $G$-spectra satisfying the given connectivity bound is closed under these operations. It thus suffices to verify the claim on the generating $\aO$-slice cells. Let $C = G_+ \wedge_H S^{m \rho_H^\aO - \epsilon} \in \tau_{\ge n}^\aO \SpG$, so $m|H|_\aO - \epsilon \ge n$.

By the Mackey double coset formula,
\[ \Phi^L C \simeq \bigvee_{g \in L \backslash G / H, \; L \le gHg^{-1}} \Phi^L \left( S^{m g\rho_H^\aO - \epsilon} \right). \]
For a valid double coset $g$, the $L$-fixed points of $g\rho_H^\aO$ have dimension $d_g$ given precisely by the number of $L$-orbits on $g H g^{-1} / g K_H g^{-1}$. By Lemma \ref{lem:orbit_bound}, $d_g = |H|_\aO / w_L$ where $w_L = [L : L \cap g K_H g^{-1}] \le |L|_\aO$. The wedge summand corresponding to $g$ is a sphere of dimension $D = m d_g - \epsilon$. We compute:
\[ D = m \frac{|H|_\aO}{w_L} - \epsilon \ge \frac{n + \epsilon}{w_L} - \epsilon = \frac{n}{w_L} + \epsilon\left(\frac{1}{w_L} - 1\right). \]

If $\epsilon = 0$, $D \ge n/w_L$. Since $D$ is an integer (the dimension of a real fixed-point subspace), it is bounded below by the ceiling of $n/w_L$: $D \ge \lceil n/w_L \rceil \ge \lfloor n/w_L \rfloor \ge \lfloor n/|L|_\aO \rfloor$.

If $\epsilon = 1$, $D \ge \frac{n+1}{w_L} - 1$. Since $D$ is an integer, $D \ge \lceil \frac{n+1}{w_L} \rceil - 1$. Write $n = q w_L + r$ via the division algorithm with remainder $0 \le r \le w_L - 1$. We have $\lfloor n/w_L \rfloor = q$. Then $\lceil \frac{n+1}{w_L} \rceil - 1 = \lceil q + \frac{r+1}{w_L} \rceil - 1 = q + \lceil \frac{r+1}{w_L} \rceil - 1$. Because $1 \le r+1 \le w_L$, we have $\lceil \frac{r+1}{w_L} \rceil = 1$. Thus, the expression exactly equals $q = \lfloor n/w_L \rfloor$, which is again $\ge \lfloor n/|L|_\aO \rfloor$. In both exhaustive cases, $\Phi^L C$ is a wedge of spheres of dimension $\ge \lfloor n/|L|_\aO \rfloor$, making it $\lfloor n/|L|_\aO \rfloor$-connective.

\textbf{Sufficiency ($\Leftarrow$):} We proceed by contradiction utilizing the exact triangle associated to the $\aO$-slice tower:
\[ F \to X \to W \to \Sigma F \]
where $F = \tau_{\ge n}^\aO X \in \tau_{\ge n}^\aO \SpG$ and $W = P_{<n}^\aO X \in (\tau_{\ge n}^\aO \SpG)^\perp$. Because $X$ is connective and $F$ is generated by $\aO$-slice cells $C$ whose geometric fixed points are universally connective (as established in the necessity direction, $\Phi^K C$ is $\lfloor n/|K|_\aO \rfloor$-connective for all $K \le G$, and $\lfloor n/|K|_\aO \rfloor \ge 0$ since $n \ge 0$), the generators of $F$ are equivariantly connective. Consequently, both $X$ and $F$ are connective spectra. Thus, $W$ sits in a cofiber sequence of connective spectra, meaning $W$ is itself bounded below. By the necessity direction, $\Phi^L F$ is $\lfloor n/|L|_\aO \rfloor$-connective. The long exact sequence of homotopy groups combined with the connectivity assumption on $X$ implies $\Phi^L W$ is also $\lfloor n/|L|_\aO \rfloor$-connective.

Assume for contradiction that $W \not\simeq \ast$. The geometric fixed points of a non-trivial bounded-below $G$-spectrum cannot be universally contractible. Thus, we may choose a subgroup $L \le G$ of minimal order such that $\Phi^L W \not\simeq \ast$. Let $c$ be the lowest integer such that $\pi_c(\Phi^L W) \neq 0$. By the connectivity bound, $c \ge q$, where $q = \lfloor n/|L|_\aO \rfloor \ge 0$.

Let $\mathcal{P}$ denote the family of proper subgroups of $L$. Because $L$ was chosen to be minimal, $\Phi^K W \simeq \ast$ for all $K \in \mathcal{P}$. The geometric fixed points of the bounded-below $L$-spectrum $E\mathcal{P}_+ \wedge W$ satisfy $\Phi^K(E\mathcal{P}_+ \wedge W) \simeq \Phi^K(E\mathcal{P}_+) \wedge \Phi^K W$. For $K \in \mathcal{P}$, this is contractible because $\Phi^K W \simeq \ast$. For $K = L$, it is contractible because $\Phi^L(E\mathcal{P}_+) \simeq (E\mathcal{P}^L)_+ \simeq \emptyset_+ \simeq \ast$. Since its geometric fixed points vanish for all subgroups, $E\mathcal{P}_+ \wedge W \simeq \ast$ by the equivariant Whitehead theorem.

The isotropy separation sequence $E\mathcal{P}_+ \wedge W \to W \to \tilde{E}\mathcal{P} \wedge W$ then implies that the natural map $W \to \tilde{E}\mathcal{P} \wedge W$ is an equivalence of $L$-spectra. Consequently, the true fixed points satisfy $W^L \simeq (\tilde{E}\mathcal{P} \wedge W)^L \simeq \Phi^L W$. We may thus choose a non-nullhomotopic map $S^c \to \Phi^L W \simeq W^L$.

We claim that for any real representation $U$ of $L$ and $\delta \in \{0, 1\}$, there is a natural isomorphism $[S^{U-\delta}, W]_L \cong [S^{\dim(U^L)-\delta}, W^L]$.
Decompose $U = U^L \oplus U'$, where $(U')^L = 0$. The unit sphere $S(U')$ is an $L$-CW complex with isotropy contained in $\mathcal{P}$. Consequently, the projection map $S(U') \times E\mathcal{P} \to S(U')$ is an $L$-equivariant equivalence. This implies $S(U')_+ \wedge E\mathcal{P}_+ \simeq S(U')_+$. Smashing $S(U')_+$ with the standard isotropy separation cofiber sequence $E\mathcal{P}_+ \to S^0 \to \tilde{E}\mathcal{P}$ immediately yields the exact triangle $S(U')_+ \wedge E\mathcal{P}_+ \to S(U')_+ \to S(U')_+ \wedge \tilde{E}\mathcal{P}$. Because the first map is an equivalence, the desired contractibility $S(U')_+ \wedge \tilde{E}\mathcal{P} \simeq \ast$ immediately follows. Smashing the standard cofiber sequence $S(U')_+ \to S^0 \to S^{U'}$ with $W \simeq W \wedge \tilde{E}\mathcal{P}$ yields an exact triangle $S(U')_+ \wedge W \to W \to S^{U'} \wedge W$, where the first term is contractible. Thus, the natural map $W \to S^{U'} \wedge W \simeq \Sigma^{U'} W$ is an equivalence. Using this, we deduce:
\begin{align*}
[S^{U-\delta}, W]_L &\cong [S^{U^L-\delta} \wedge S^{U'}, \Sigma^{U'} W]_L \\
&\cong [S^{U^L-\delta}, W]_L \\
&\cong [S^{\dim(U^L)-\delta}, W^L].
\end{align*}

We define the representation $U$ and parameter $\delta$ in two exhaustive cases to construct an equivariant lift:
\begin{enumerate}
    \item \textbf{Case $c > q$:} Let $U = c \rho_L^\aO$ and $\delta = 0$. The $L$-fixed points of the transitive permutation representation $\rho_L^\aO = \mR[L/K_L]$ consist solely of the constant functions, so $\dim(U^L) = c(1) = c$.
    Under the isomorphism above, the non-nullhomotopic map $S^c \to W^L$ lifts to a non-trivial $L$-equivariant map $S^{c \rho_L^\aO} \to W$. By the induction-restriction adjunction, this canonically corresponds to a non-trivial $G$-equivariant map $f: C \to W$, where $C = G_+ \wedge_L S^{c \rho_L^\aO}$. Its slice dimension is $c |L|_\aO$. Since $c > q = \lfloor n/|L|_\aO \rfloor$, we have $c \ge \lfloor n/|L|_\aO \rfloor + 1 > n/|L|_\aO$, ensuring $c |L|_\aO > n$. Because $c > q \ge 0$, we have $c \ge 1 > 0$, ensuring $C$ is a valid $\aO$-slice cell of dimension $\ge n$. Thus $C \in \tau_{\ge n}^\aO \SpG$.

    \item \textbf{Case $c = q$:} Let $U = (q+1)\rho_L^\aO$ and $\delta = 1$. Then $\dim(U^L) - \delta = (q+1)(1) - 1 = q = c$.
    The non-nullhomotopic map $S^c \to W^L$ lifts to a non-trivial $L$-equivariant map $S^{(q+1)\rho_L^\aO - 1} \to W$. By the induction-restriction adjunction, this uniquely yields a non-trivial $G$-equivariant map $f: C \to W$, where $C = G_+ \wedge_L S^{(q+1)\rho_L^\aO - 1}$. Its slice dimension is $(q+1)|L|_\aO - 1$. Write $n = q|L|_\aO + r$ with remainder $0 \le r \le |L|_\aO - 1$. We have $(q+1)|L|_\aO - 1 = q|L|_\aO + |L|_\aO - 1 \ge q|L|_\aO + r = n$. Since $n \ge 0$ and $q \ge 0$, we have $q+1 \ge 1 > 0$. Hence, $C$ is a valid $\aO$-slice cell and $C \in \tau_{\ge n}^\aO \SpG$.
\end{enumerate}

In both configurations, we found an $\aO$-slice cell $C \in \tau_{\ge n}^\aO \SpG$ admitting a non-nullhomotopic map $f: C \to W$. However, $W \in (\tau_{\ge n}^\aO \SpG)^\perp$ fundamentally resides in the right-orthogonal complement of $\tau_{\ge n}^\aO \SpG$, demanding that any map from $C$ to $W$ be universally nullhomotopic. This constitutes a strict contradiction. Therefore, we must have $W \simeq \ast$, implying $X \simeq F \in \tau_{\ge n}^\aO \SpG$.
\end{proof}


\end{solution}
\end{document}
