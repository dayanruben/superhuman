\documentclass{article}
\usepackage{geometry}
\usepackage{graphicx, url, hyperref, amsmath, amssymb, mathtools, comment, xcolor}
\usepackage{amsthm}
\usepackage[most]{tcolorbox}
\tcbuselibrary{skins, breakable}
\usepackage[shortlabels]{enumitem}
\usepackage{mathscinet} % for \Dbar

\usepackage{booktabs}
\usepackage{tabularx}

% Geometry settings
\geometry{margin=1.5in}

% --- Theorems and Styles ---
\theoremstyle{plain}
\newtheorem{theorem}{Theorem}
\newtheorem{proposition}{Proposition}
\newtheorem{lemma}{Lemma}
\newtheorem{claim}{Claim}

\theoremstyle{definition}
\newtheorem*{example}{Example}
\newtheorem{remark}{Remark}[section]% Theorem environments

\theoremstyle{definition}
\newtheorem{definition}{Definition}
% --- Custom Commands ---
\newcommand{\Aletheia}{\emph{Aletheia}}
\renewcommand{\comment}[1]{\marginpar{{\tiny{#1}\normalfont\par}}}
\newcommand{\tony}[1]{{\color{red}Tony: #1}}

% --- Defined Colors ---
\definecolor{boxblue}{RGB}{0, 0, 150}
\definecolor{boxback}{RGB}{245, 245, 255}

% --- The Problem Environment ---
\newtcolorbox{problem}[1]{%
    colback=boxback,
    colframe=boxblue,
    fonttitle=\bfseries\large,
    title={#1},
    sharp corners,
    enhanced jigsaw, % Better frame handling for page breaks than just 'enhanced'
    breakable,       % <--- Allows the box to split across pages
    attach boxed title to top left={yshift=-2mm, xshift=2mm},
    boxed title style={colframe=boxblue, colback=boxblue},
    before skip=15pt plus 2pt,
    after skip=15pt plus 2pt,
    top=10pt, bottom=10pt, left=10pt, right=10pt
}

\newtcolorbox{solution}[1]{%
    colback=white,
    colframe=boxblue,
    fonttitle=\bfseries\large,
    title={#1},
    sharp corners,
    enhanced jigsaw, % Better frame handling for page breaks than just 'enhanced'
    breakable,       % <--- Allows the box to split across pages
    attach boxed title to top left={yshift=-2mm, xshift=2mm},
    boxed title style={colframe=boxblue, colback=boxblue},
    before skip=15pt plus 2pt,
    after skip=15pt plus 2pt,
    top=10pt, bottom=10pt, left=10pt, right=10pt
}



% Define custom commands for readability
\newcommand{\Q}{\mathbb{Q}}
\newcommand{\VV}{\mathbb{V}}
\newcommand{\frR}{\mathfrak{R}}
\newcommand{\co}{\colon}
\DeclareMathOperator{\Gr}{Gr}
\DeclareMathOperator{\aug}{aug}



\date{}


\begin{document}
\emph{The prompts and responses below have been edited only for
formatting compatibility.}

\begin{problem}{User Prompt (ver 1)}
Is it true that if $a_1<a_2<\cdots$ is a sequence of integers with\[\liminf a_n^{1/2^n}>1\]then\[\sum_{n=1}^\infty \frac{1}{a_na_{n+1}}\]is irrational?
\end{problem}


\begin{solution}{Model Response:}
\begin{theorem}
Let $(a_n)_{n \ge 1}$ be a strictly increasing sequence of integers such that
\[ \liminf_{n \to \infty} a_n^{1/2^n} > 1. \]
Then the infinite sum
\[ S = \sum_{n=1}^\infty \frac{1}{a_n a_{n+1}} \]
is irrational.
\end{theorem}


\paragraph{Proof.}
\paragraph{1. Growth Estimates Based on Hypothesis}
Let $\lambda = \liminf_{n \to \infty} a_n^{1/2^n}$. By hypothesis, $\lambda > 1$. Select a constant $c$ such that $1 < c <1 < c < \lambda$. By the definition of limit inferior, there exists an integer $N_0$ such that for all $n \ge N_0$:
\[ a_n^{1/2^n} > c \implies a_n > c^{2^n}. \]
Let $u_n = \ln a_n$. Taking the logarithm, we have $u_n > 2^n \ln c$.
Let $T_n = \sum_{i=1}^n u_i$. Bounding the sum from below:
\[ T_n > \sum_{i=N_0}^n 2^i \ln c = \ln c (2^{n+1} - 2^{N_0}). \]
Let $\alpha = \ln c > 0$. For sufficiently large $n$, we have:
\[ T_n > \alpha 2^n. \tag{1} \]
Since $a_n$ is a strictly increasing sequence of integers, $a_n \ge 1$ for all sufficiently large $n$. We assume $n \ge N_0$ throughout the remaining argument.

\paragraph{2. Assumption of Rationality and Lower Bound}
Assume, for the sake of contradiction, that $S$ is rational. Let $S = \frac{p}{q}$ for positive integers $p, q$.
Define the $n$-th partial sum $S_n$ and the remainder $r_n$:
\[ S_n = \sum_{k=1}^n \frac{1}{a_k a_{k+1}}, \qquad r_n = S - S_n = \sum_{k=n+1}^\infty \frac{1}{a_k a_{k+1}}. \]
Since the terms are positive, $S_n <n < S$, so $r_n > 0$.
Consider the denominators in $S_n$. Each term is of the form $\frac{1}{a_k a_{k+1}}$. Notice that $a_k a_{k+1}$ divides the product $P_{n+1} = \prod_{i=1}^{n+1} a_i$.
Let $L_n = \operatorname{lcm}(a_1 a_2, \dots, a_n a_{n+1})$. Then $L_n \le P_{n+1}$.
Writing $S_n = \frac{M_n}{L_n}$ for some integer $M_n$, we have:
\[ r_n = \frac{p}{q} - \frac{M_n}{L_n} = \frac{p L_n - q M_n}{q L_n}. \]
Since $r_n > 0$, the numerator is a positive integer, so $p L_n - q M_n \ge 1$.
Thus:
\[ r_n \ge \frac{1}{q L_n} \ge \frac{1}{q P_{n+1}}. \tag{2} \]
Using the relation $P_{n+1} = a_{n+1} P_n$, we write this as:
\[ r_n \ge \frac{1}{q a_{n+1} P_n}. \]

\paragraph{3. Universal Upper Bound for Remainder}
We now bound $r_{n+1} = \sum_{k=n+2}^\infty \frac{1}{a_k a_{k+1}}$.
Since $(a_k)$ is strictly increasing, $a_{k+1} > a_k$, so $\frac{1}{a_k a_{k+1}} < \frac{1}{a_k^2}$.
Thus, $r_{n+1} < \sum_{k=n+2}^\infty \frac{1}{a_k^2}$.

Since terms may be clustered, we cannot assume $a_{k+1} \gg a_k$. Instead, we partition the indices $k \ge n+2$ into sets $I_j$ based on powers of 2 relative to $a_{n+2}$:
\[ I_j = \{ k \ge n+2 \mid 2^j a_{n+2} \le a_k < 2^{j+1} a_{n+2} \}, \quad j = 0, 1, 2, \dots \]
For any $k \in I_j$, we have $a_k \ge 2^j a_{n+2}$, so $\frac{1}{a_k^2} \le \frac{1}{(2^j a_{n+2})^2}$.
We estimate the size $|I_j|$. For any $k \in I_j$, we have $a_k < 2^{j+1} a_{n+2}$.
From the growth condition, $a_k > c^{2^k}$. Combining these:
\[ c^{2^k} < 2^{j+1} a_{n+2} \implies 2^k \ln c < \ln c < \ln(2^{j+1}) + \ln a_{n+2}. \]
\[ k < \log_2 \left( \frac{(j+1)\ln 2 + \ln a_{n+2}}{\ln c} \right). \]
Since the indices $k$ are positive integers, the number of such indices in $I_j$ is bounded by the maximum possible index.
Let $L = \ln a_{n+2}$. For sufficiently large $n$ (and thus large $L$), the dominant term is $\ln L$.
There exists a constant $C > 0$ such that $|I_j| \le C (\ln L + j)$.
Substituting this into the sum:
\[ \sum_{k=n+2}^\infty \frac{1}{a_k^2} = \sum_{j=0}^\infty \sum_{k \in I_j} \frac{1}{a_k^2} \le \sum_{j=0}^\infty \frac{C(\ln L + j)}{(2^j a_{n+2})^2}. \]
\[ r_{n+1} < \frac{C}{a_{n+2}^2} \left( \ln L \sum_{j=0}^\infty \frac{1}{4^j} + \sum_{j=0}^\infty \frac{j}{4^j} \right). \]
The series converge to finite constants. Thus, for some constant $K$:
\[ r_{n+1} < \frac{K \ln L}{a_{n+2}^2} = \frac{K \ln \ln a_{n+2}}{a_{n+2}^2}. \tag{3} \]

\paragraph{4. Deriving the Recurrence Relation}
Recall the identity $r_n = \frac{1}{a_{n+1} a_{n+2}} + r_{n+1}$.
Combining the lower bound (2) and the upper bound (3):
\[ \frac{1}{q a_{n+1} P_n} \le \frac{1}{a_{n+1} a_{n+2}} + \frac{K \ln \ln a_{n+2}}{a_{n+2}^2}. \]
Multiplying through by $a_{n+1} a_{n+2}$:
\[ \frac{a_{n+2}}{q P_n} \le 1 + \frac{a_{n+1}}{a_{n+2}} K \ln \ln a_{n+2}. \]
Since the sequence is strictly increasing, $a_{n+1} < a_{n+2}$, implying $\frac{a_{n+1}}{a_{n+2}} < 1$.
Thus:
\[ \frac{a_{n+2}}{q P_n} < 1 + K \ln \ln a_{n+2}. \]
Rearranging to isolate $a_{n+2}$:
\[ a_{n+2} < q P_n (1 + K \ln \ln a_{n+2}). \]
Taking the natural logarithm ($u_n = \ln a_n$ and $\ln P_n = T_n$):
\[ u_{n+2} < \ln q + T_n + \ln(1 + K \ln u_{n+2}). \]
Since $a_n \to \infty$, $u_{n+2} \to \infty$. Thus for any $\epsilon > 0$, there exists $N_\epsilon$ such that for all $n \ge N_\epsilon$:
\[ \ln(1 + K \ln u_{n+2}) < \epsilon u_{n+2}. \]
Using this bound:
\[ u_{n+2} < T_n + \ln q + \epsilon u_{n+2}. \]
\[ (1 - \epsilon) u_{n+2} < T_n + \ln q. \]
Dividing by $1-\epsilon$:
\[ u_{n+2} < \frac{1}{1-\epsilon} T_n + \frac{\ln q}{1-\epsilon}. \]
Substitute $u_{n+2} = T_{n+2} - T_{n+1}$:
\[ T_{n+2} - T_{n+1} < \frac{1}{1-\epsilon} T_n + C_\epsilon. \]
Rearranging gives a linear recurrence inequality:
\[ T_{n+2} < T_{n+1} + \frac{1}{1-\epsilon} T_n + C_\epsilon. \]

\paragraph{5. Contradiction}
Let us analyze the growth rate implied by this recurrence.
Consider the sequence $X_n$ satisfying the corresponding homogeneous equality $X_{n+2} = X_{n+1} + \frac{1}{1-\epsilon} X_n$.
The characteristic equation is $x^2 - x - \frac{1}{1-\epsilon} = 0$.
The positive root is:
\[ \rho(\epsilon) = \frac{1 + \sqrt{1 + \frac{4}{1-\epsilon}}}{2}. \]
Since $T_n$ consists of positive terms, standard comparison theorems for linear recurrences imply that for sufficiently large $n$:
\[ T_n < T_n < C \cdot (\rho(\epsilon))^n. \]
We now choose $\epsilon$ sufficiently small to force a contradiction with the known lower bound.
Let $\epsilon = 0.05$. Then $\frac{1}{1-\epsilon} \approx 1.053$.
\[ \rho(0.05) = \frac{1 + \sqrt{1 + 4.21}}{2} = \frac{1 + \sqrt{5.21}}{2} \approx \frac{1 + 2.28}{2} = 1.64. \]
Crucially, $\rho(0.05)< 1.7 < 2$.
Thus, if $S$ is rational, then $T_n = O(1.7^n)$.

However, from Step 1 (Inequality 1), we established that the condition $\liminf a_n^{1/2^n} > 1$ implies:
\[ T_n > \alpha 2^n. \]
Combining these bounds yields:
\[ \alpha 2^n <n < C (1.7)^n \implies \alpha <implies \alpha < C \left( \frac{1.7}{2} \right)^n. \]
Since $\frac{1.7}{2}< 1$, the right hand side approaches 0 as $n \to \infty$.
This implies $\alpha \le 0$.
But $\alpha = \ln c > 0$.
This is a contradiction.

Therefore, our initial assumption that $S$ is rational must be false.

\paragraph{Conclusion}
The sum $S = \sum_{n=1}^\infty \frac{1}{a_n a_{n+1}}$ is irrational.
\end{solution}

\begin{problem}{User Prompt (ver 2)}
Is it true that if $a_1<a_2<\cdots$ is a sequence of integers with\[\liminf a_n^{1/2^n}>1\]then\[\sum_{n=1}^\infty \frac{1}{a_na_{n+1}}\]is irrational?
\end{problem}

\begin{solution}{Model Response:}
The problem asks us to determine the truth of the statement: if $(a_n)$ is a strictly increasing sequence of integers with $\liminf a_n^{1/2^n} > 1$, then $\sum \frac{1}{a_n a_{n+1}}$ is irrational.

\section*{Executive Summary}
Status: Complete Solution

We prove that the statement is true. The argument proceeds by contradiction. We first reduce the problem to the case where all $a_n \ge 2$ without loss of generality. Assuming the series sums to a rational number $p/q$, we construct a sequence of integers $Z_n = q P_{n+1} R_n$, where $R_n$ is the series remainder and $P_{n+1}$ is the partial product of terms. The properties of $(a_n)$ ensure $Z_n$ is a positive integer for all $n$. We derive an upper bound for $Z_n$ that leads to the inequality $a_{n+2} < 2q P_n$ for sufficiently large $n$. This recurrence implies that $\ln P_n$ grows no faster than the Fibonacci sequence (order $\phi^n$). However, the given condition $\liminf a_n^{1/2^n} > 1$ implies that $\ln P_n$ grows at least as $2^n$. Since $2 > \phi$, these growth rates are incompatible, establishing the contradiction.

\section*{Polished Solution}
Let $(a_n)_{n=1}^\infty$ be a strictly increasing sequence of integers satisfying $\liminf_{n \to \infty} a_n^{1/2^n} > 1$. We prove that the sum of the series $\sum_{n=1}^\infty \frac{1}{a_n a_{n+1}}$ is an irrational number.

\subsection*{Irrationality of the Series $\sum_{n=1}^\infty \frac{1}{a_n a_{n+1}}$ Under Rapid Growth Conditions}

\subsubsection*{Introduction}

We analyze the arithmetic nature of the convergent series
\[ S = \sum_{n=1}^\infty \frac{1}{a_n a_{n+1}}, \]
given that the integer sequence $(a_n)_{n=1}^\infty$ grows doubly exponentially. Specifically, we assume:
\begin{equation} \label{eq:growth_condition}
\liminf_{n \to \infty} a_n^{1/2^n} > 1.
\end{equation}
We demonstrate that this growth condition forces the sum $S$ to be irrational.


\subsubsection*{Proof of Irrationality}

\begin{theorem}
Let $(a_n)_{n=1}^\infty$ be a strictly increasing sequence of integers such that
\[ \liminf_{n \to \infty} a_n^{1/2^n} > 1. \]
Then
\[ S = \sum_{n=1}^\infty \frac{1}{a_n a_{n+1}} \notin \mathbb{Q}. \]
\end{theorem}

\begin{proof}
Let $\lambda = \liminf_{n \to \infty} a_n^{1/2^n}$. By hypothesis, $\lambda > 1$. Let $c$ be a constant such that $1 < c < \lambda$. By the definition of limit inferior, there exists $N_0$ such that for all $n \ge N_0$, $a_n^{1/2^n} > c$, or equivalently $a_n > c^{2^n}$. Since $c > 1$, $a_n \to \infty$. Because $(a_n)$ is strictly increasing, the terms eventually become positive integers. Thus, there exists $N_1 \ge N_0$ such that $a_n \ge 2$ for all $n \ge N_1$.

Note that the rationality of $S$ depends only on the tail of the series. Let $S = \sum_{n=1}^{N_1-1} \frac{1}{a_n a_{n+1}} + S_{\text{tail}}$. The finite initial sum is rational. Thus, $S \in \mathbb{Q} \iff S_{\text{tail}} \in \mathbb{Q}$.
We shift the indices of the sequence by defining $a'_n = a_{n + N_1 - 1}$ for $n \ge 1$. The condition (\ref{eq:growth_condition}) is invariant under this shift (with a modified exponent factor that does not affect the strict inequality $\lambda' > 1$). Without loss of generality, we proceed assuming $a_n \ge 2$ and $a_n > c^{2^n}$ for all $n \ge 1$.
Convergence is guaranteed by comparison with the geometric series $\sum c^{-2^n}$, which converges absolutely.

We begin by characterizing the relative growth of the terms.

\begin{lemma} \label{lem:divergence}
Under the hypotheses, $\lim_{n \to \infty} \frac{a_{n+2}}{a_n} = \infty$.
\end{lemma}

\begin{proof}
Suppose the limit is not infinity. Then there exist a constant $M > 0$ and a subsequence $(n_k)$ such that $a_{n_k+2} \le M a_{n_k}$. Rearranging yields $a_{n_k} \ge M^{-1} a_{n_k+2}$. Raising both sides to the power $1/2^{n_k}$:
\[ a_{n_k}^{1/2^{n_k}} \ge M^{-1/2^{n_k}} \left( a_{n_k+2}^{1/2^{n_k+2}} \right)^{2^{n_k+2}/2^{n_k}}. \]
Substituting $2^{n_k+2}/2^{n_k} = 4$:
\[ a_{n_k}^{1/2^{n_k}} \ge M^{-1/2^{n_k}} \left( a_{n_k+2}^{1/2^{n_k+2}} \right)^4. \]
Taking the limit inferior as $k \to \infty$:
\begin{enumerate}
    \item $\liminf a_{n_k}^{1/2^{n_k}} \ge \liminf a_n^{1/2^n} = \lambda$.
    \item $\lim M^{-1/2^{n_k}} = 1$.
    \item $\liminf \left( a_{n_k+2}^{1/2^{n_k+2}} \right)^4 \ge \lambda^4$, since any subsequence of $a_n^{1/2^n}$ has limit inferior at least $\lambda$.
\end{enumerate}
Thus, $\lambda \ge \lambda^4$. Since $\lambda > 1$, dividing by $\lambda$ yields $1 \ge \lambda^3$, which implies $\lambda \le 1$. This contradicts $\lambda > 1$. Therefore, the ratio must tend to infinity.
\end{proof}

From Lemma \ref{lem:divergence}, $\frac{a_n}{a_{n+2}} \to 0$. Hence, there exists an integer $N^* \ge 1$ such that for all $n \ge N^*$:
\begin{equation} \label{eq:ratio_bound}
\frac{a_n}{a_{n+2}} \le \frac{1}{4}.
\end{equation}

Assume for the sake of contradiction that $S = \frac{p}{q}$ for integers $p, q \ge 1$.
Define the partial product $P_n = \prod_{i=1}^n a_i$.
Let $R_n = \sum_{k=n+1}^\infty \frac{1}{a_k a_{k+1}}$.
Define $Z_n = q P_{n+1} R_n$.
Writing $R_n = S - \sum_{k=1}^n \frac{1}{a_k a_{k+1}}$, we have:
\[ Z_n = q P_{n+1} \left( \frac{p}{q} - \sum_{k=1}^n \frac{1}{a_k a_{k+1}} \right) = p P_{n+1} - q \sum_{k=1}^n \frac{P_{n+1}}{a_k a_{k+1}}. \]
Consider the term $T_k = \frac{P_{n+1}}{a_k a_{k+1}}$ for $1 \le k \le n$. Since $a_i$ are distinct integers (strictly increasing), the factors $a_k$ and $a_{k+1}$ are distinct elements in the set $\{a_1, \dots, a_{n+1}\}$. Thus, their product divides $P_{n+1}$. Consequently, $T_k \in \mathbb{Z}$ for all $k$, implying $Z_n \in \mathbb{Z}$.
Since $a_n \ge 2$, all terms in the series are positive, so $R_n > 0$. Since $q, P_{n+1} > 0$, we have $Z_n > 0$. As $Z_n$ is an integer:
\begin{equation} \label{eq:Zn_pos}
Z_n \ge 1.
\end{equation}

We now bound $R_n$. Let $u_k = \frac{1}{a_k a_{k+1}}$.
For $k \ge N^*$, the ratio of consecutive series terms is:
\[ \frac{u_{k+1}}{u_k} = \frac{a_k a_{k+1}}{a_{k+1} a_{k+2}} = \frac{a_k}{a_{k+2}} \le \frac{1}{4}. \]
Using this geometric bound for $n \ge N^*$:
\[ R_n = \sum_{j=1}^\infty u_{n+j} \le u_{n+1} \sum_{j=0}^\infty \left( \frac{1}{4} \right)^j = u_{n+1} \frac{4}{3} = \frac{4}{3 a_{n+1} a_{n+2}}. \]
Substituting this into the definition of $Z_n$:
\[ Z_n \le q P_{n+1} \frac{4}{3 a_{n+1} a_{n+2}} = \frac{4 q P_n}{3 a_{n+2}}. \]
Applying (\ref{eq:Zn_pos}):
\[ 1 \le \frac{4 q P_n}{3 a_{n+2}} \implies a_{n+2} \le \frac{4}{3} q P_n. \]
Since $4/3 < 2$, we strictly have:
\begin{equation} \label{eq:upper_growth}
a_{n+2} < 2 q P_n.
\end{equation}

This inequality relates the term $a_{n+2}$ to the cumulative product $P_n$. We analyze the growth of $P_n$.
From $P_{n+2} = P_{n+1} a_{n+2}$, we deduce:
\[ P_{n+2} < 2 q P_n P_{n+1}. \]
Let $L_n = \ln P_n$. Since $P_n \ge 2^n$, $L_n > 0$.
\[ L_{n+2} < L_{n+1} + L_n + \ln(2q). \]
Let $K = \ln(2q)$. Define an auxiliary sequence $(x_n)$ by $x_{n+2} = x_{n+1} + x_n + K$ with initial conditions $x_{N^*} = L_{N^*}, x_{N^*+1} = L_{N^*+1}$.
By simple induction, $L_n \le x_n$ for all $n \ge N^*$.
Let $y_n = x_n + K$. Then $y_{n+2} = y_{n+1} + y_n$.
Thus $y_n$ follows the Fibonacci recurrence. Standard solutions give $y_n = A \phi^n + B \psi^n$, where $\phi = \frac{1+\sqrt{5}}{2}$.
Since $|\psi| < 1$, for large $n$, $y_n \sim A \phi^n$. Consequently, there exists $C_1 > 0$ such that:
\begin{equation} \label{eq:bound_fib}
L_n \le C_1 \phi^n.
\end{equation}

Now we contrast this with the lower bound implied by the hypothesis.
Recall $a_n > c^{2^n}$ for $n \ge N^*$ (renormalized index), with $c > 1$.
\[ P_n = \prod_{k=1}^n a_k > \left( \prod_{k=1}^{N^*-1} a_k \right) \prod_{k=N^*}^n c^{2^k}. \]
The exponent sum is $\sum_{k=N^*}^n 2^k = 2^{n+1} - 2^{N^*}$.
So $P_n > \text{Const} \cdot c^{2^{n+1}}$.
Taking natural logarithms:
\[ L_n > 2^{n+1} \ln c + C_{\text{const}}. \]
Since $\ln c > 0$, for sufficiently large $n$, there exists $C_2 > 0$ such that:
\begin{equation} \label{eq:bound_exp}
L_n > C_2 2^n.
\end{equation}

Comparing (\ref{eq:bound_fib}) and (\ref{eq:bound_exp}):
\[ C_2 2^n < L_n \le C_1 \phi^n. \]
Dividing by $\phi^n$:
\[ C_2 \left( \frac{2}{\phi} \right)^n < C_1. \]
Observe that $\phi = \frac{1+\sqrt{5}}{2} \approx 1.618$, so $2 > \phi$. Thus $\frac{2}{\phi} > 1$.
Consequently, $\lim_{n \to \infty} \left( \frac{2}{\phi} \right)^n = \infty$.
The inequality implies that a sequence tending to infinity is strictly bounded by the constant $C_1/C_2$, which is absurd.

Thus, the initial assumption that $S$ is rational leads to a contradiction. We conclude that $S$ must be irrational.
\end{proof}
\end{solution}

\begin{problem}{User Prompt (Gemini Deep Think)}
Find a meaningful, interesting generalization of this theorem or its proof.

\textbf{Theorem}
Fix $d\ge2$, and let $c_d$ be the unique positive real root  of $x^{d}-x^{d-1}-1=0$. Suppose a strictly increasing sequence $\{a_n\}$ of integers satisfy
$$ \liminf_{n \to \infty} a_n^{1/b^n} > 1$$
for some $b>c_d$. Then the infinite sum
$$ S = \sum_{n=1}^\infty \frac{1}{a_n a_{n+1}\cdots a_{n+d-1}} $$
is irrational.

\textbf{Proof}
For two functions $f$ and $g$, we write $f\precsim g$ when there exists some constant $C>0$ satisfying $f\le Cg$. Assume for contradiction that $S=p/q$.  We may assume there is some constant $c>1$ such that $a_n>c^{b^n}$ for all $n$. We set
$$P_n:=\prod_1^n a_i,\quad  S_{n} = \sum_{m=1}^n \frac{1}{a_m a_{m+1}\cdots a_{m+d-1}} .$$
Since $S>S_n$ for all $n$, we have
$$r_{n} := S - S_n=\frac{p}{q}-\frac{*}{P_{n+d-1}} \ge \frac1{qP_{n+d-1}}.$$
Let us define
$$ I_j(n):= \{ k \ge n+1 \mid 2^j a_{n+1}^{d} \le a_k^{d} < 2^{j+1} a_{n+1}^{d} \}, \quad j = 0, 1, 2, \dots $$
so that each $k \in I_j(n)$ satisfies
$$c^{d\cdot b^k}<a_k^d < 2^{j+1} a_{n+1}^d.$$
The cardinality of $I_j(n)$ can be bounded from above as
$$
|I_{j}(n)|\le \max_{k\in I_{j}(n)}k \le \log_{b}\left(\frac{\log_{c}(2^{j+1}a_{n+1}^d)}{d}\right)\precsim \log j+\log\log a_{n+1}.
$$
It follows that
$$ r_{n} < \sum_{k=n+1}^\infty \frac{1}{a_k^{d}} = \sum_{j=0}^\infty \sum_{k \in I_j(n)} \frac{1}{a_k^{d}}
\le
\sum_{j=0}^\infty  \frac{| I_j(n)|}{2^j a_{n+1}^{d}}
\precsim \sum_{j=0}^\infty \frac{\log j+\log\log a_{n+1} }{2^j a_{n+1}^d}\precsim \frac{ \log \log a_{n+1}}{a_{n+1}^{d}}.$$
Combining the lower and the upper bounds for $r_n$ and $r_{n+1}$, we have
\begin{align*}\frac{1}{q P_{n+d-1}} &\le r_{n}= \sum_{j=1}^{d-1}\frac{1}{a_{n+j} \cdots a_{j+n+d-1}} + r_{n+d-1}\\
&\precsim\frac{d-1}{a_{n+1}\cdots a_{n+d}} + \frac{\log\log a_{n+d}}{a_{n+d}^d}\le \frac{\log\log a_{n+d}}{a_{n+1}\cdots a_{n+d}}. \end{align*}
Fix an arbitrarily small $\epsilon>0$. Multiplying both sides by $P_{n+d}$, we obtain that
$$a_{n+d}=P_{n+d}/P_{n+d-1}\precsim P_{n}\log\log a_{n+d}
\precsim P_{n}a_{{n+d}}^\epsilon=P_{n}(P_{n+d}/P_{n+d-1})^\epsilon.$$

It follows that we have $P_{n+d}\precsim P_{n+d-1}P_{n}^{1/(1-\epsilon)}$. The sequence $u_n:=\log_{c} P_n$ satisfies the recursive inequality
$$u_{n+d}\le u_{n+d-1}+\frac1{1-\epsilon} u_{n} + K$$
for some constant $K=K(\epsilon)>0$. For a sufficiently small $\epsilon$, the equation $$x^d=x^{d-1}+\frac{1}{1-\epsilon}$$
has exactly one positive real root $s=s(d,\epsilon)$, slightly larger than $c_d$, and one or zero negative real root, depending on the parity of $d$; moreover, the negative real root, if exists, belong to $(-1,0)$. By choosing a sufficiently small $\epsilon$ satisfying $s=s(d,\epsilon)<b$,  we have that $u_n\precsim s^n$. This contradicts the estimate $\log_c P_n\ge \log_{c} a_{n}\ge b^n$. **QED**
\end{problem}

\begin{solution}{Model Response:}
A meaningful and natural generalization of this theorem extends the result to sums of reciprocals of products with **arbitrary integer weights**.

The original theorem relies on a product of $d$ consecutive terms, effectively assigning a weight of $1$ to every integer in a "sliding window." By generalizing these exponents, we can quantify how the density or sparsity of the denominator affects the growth rate required for the series to be irrational.

\subsection*{Theorem (Generalized)}

Fix an integer $d \ge 1$ and a tuple of non-negative integers $\mathbf{w} = (w_0, w_1, \dots, w_{d-1})$ with $w_{d-1} \ge 1$. Let $W = \max(w_0, \dots, w_{d-1})$.

Let $c_{\mathbf{w}}$ be the unique real root greater than 1 of the polynomial:

$$ P(x) = (x-1)\sum_{j=0}^{d-1} w_j x^j - W x^{d-1} $$

Suppose a strictly increasing sequence of integers $\{a_n\}$ satisfies

$$ \liminf_{n \to \infty} a_n^{1/b^n} > 1 $$

for some $b > c_{\mathbf{w}}$. Then the infinite sum

$$ S = \sum_{n=1}^\infty \frac{1}{a_n^{w_0} a_{n+1}^{w_1} \cdots a_{n+d-1}^{w_{d-1}}} $$

is irrational.

---

\subsection*{Proof Sketch}

The generalization follows the "gap principle" strategy used in the original proof, balancing the decay of the tail against the growth of the common denominator.

1. \textbf{The Common Denominator (LCM):}

    Assume for contradiction that $S = p/q$. Let $S_n$ be the $n$-th partial sum. The denominators of the terms in $S_n$ involve integers up to $a_{n+d-1}$.

    Because the window shifts by 1 at each step, any specific integer $a_k$ appears in $d$ consecutive terms with exponents $w_{d-1}, \dots, w_0$. When taking the Least Common Multiple (LCM) to form the common denominator $Q_n$, the exponent of $a_k$ is bounded by the maximum weight $W$.

    $$ Q_n \lesssim \prod_{k=1}^{n+d-1} a_k^W \approx P_{n+d-1}^W $$


2. \textbf{The Tail Estimate:}

    The remainder $r_n = S - S_n$ is dominated by the first term of the tail (index $n+1$).

    $$ r_n \approx \frac{1}{a_{n+1}^{w_0} a_{n+2}^{w_1} \cdots a_{n+d}^{w_{d-1}}} $$

3. \textbf{The Inequality:}

    For $S$ to be rational, the error $r_n$ cannot be "too small" relative to the denominator $Q_n$. Specifically, $r_n \ge 1/(q Q_n)$. This implies $Q_n r_n \gtrsim 1$. Substituting the bounds:

    $$ P_{n+d-1}^W \cdot \frac{1}{\prod_{j=0}^{d-1} a_{n+1+j}^{w_j}} \gtrsim 1 \implies \prod_{j=0}^{d-1} a_{n+1+j}^{w_j} \lesssim P_{n+d-1}^W $$

    Taking logarithms (assuming $\log a_k \sim b^k$) and dividing by $b^{n+1}$:

    $$ \sum_{j=0}^{d-1} w_j b^j \lesssim W \frac{b^{d-1}}{b-1} $$

    Rearranging this inequality yields the polynomial $P(b) \le 0$. If $b > c_{\mathbf{w}}$ (the root), this inequality is violated for large $n$, proving irrationality.

---

\textbf{Why this is Meaningful}

1. \textbf{It Unifies "Dense" and "Sparse" Problems
}
\textbf{The Original Case:} If $\mathbf{w}=(1, \dots, 1)$, then $W=1$. The sum is $\frac{x^d-1}{x-1}$.

    $$ (x-1)\frac{x^d-1}{x-1} - 1\cdot x^{d-1} = x^d - x^{d-1} - 1 = 0 $$

    This recovers the original constant $c_d$.

\textbf{The "Gap" Case:} Consider $S = \sum \frac{1}{a_n a_{n+2}}$. Here $\mathbf{w}=(1, 0, 1)$ and $W=1$. The polynomial becomes:

    $$ (x-1)(1+x^2) - x^2 = x^3 - 2x^2 + x - 1 = 0 $$

    The root is $c \approx 1.75$, which is larger than the $c \approx 1.46$ required for the consecutive block $a_n a_{n+1} a_{n+2}$. This quantifies the intuition that **"missing terms" weaken the convergence**, forcing the sequence $a_n$ to grow faster to guarantee irrationality.


2. \textbf{It Handles Heavy Weights}

\textbf{Sylvester-Type Series:} If $d=1$ and $w_0=1$, the polynomial is $(x-1)(1) - 1 = x-2$, recovering the classic result that $\sum 1/a_n$ is irrational if $a_{n+1} \sim a_n^2$ (limit $b=2$).

If we increase the weight of the \textit{last} term (e.g., $1/(a_n a_{n+1}^2)$), the root increases, making irrationality harder to prove. This is because the "cost" of the LCM ($W=2$) rises, while the benefit to the tail decay is limited to the future terms.


3. \textbf{Universal Irrationality for Double-Exponential Sequences}

Since $c_d \to 1$ as $d \to \infty$ for the uniform case, this generalization implies a powerful corollary:

For \textbf{any} sequence $\{a_n\}$ with double-exponential growth (any $b > 1$), there exists a sufficiently large block size $d$ such that $\sum \frac{1}{a_n \cdots a_{n+d-1}}$ is irrational.\end{solution}
\end{document}
